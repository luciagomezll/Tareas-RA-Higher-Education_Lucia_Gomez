\documentclass[11pt,a4paper]{article}
\usepackage [latin1]{inputenc}
\usepackage[english]{babel}
\usepackage{amsmath}
\usepackage{amsfonts}
\usepackage{amssymb}
\usepackage{graphicx}
\usepackage[left=4cm,right=3cm,top=2.5cm,bottom=2.5cm]{geometry}
\usepackage{indentfirst}

\title{Legal Framework for the creation of higher-education institutions}
\author{Lucia Gomez Lactahuamani}
\date{}
\begin{document}
\maketitle

Before 1995, Law N� 23733, University Law, regulated the existence of public universities and private non-profit universities. It stated that universities were created only by law, with prior accreditation of their need, the availability of qualified staff and the resources to ensure the efficiency of services. Likewise, it established that universities could not have subsidiaries or annexes; exceptionally, universities could create faculties, within the departmental scope of the university.

In 1995, through Law N� 26439, the National Council for the Authorization of the Functioning of Universities (CONAFU) was established, an autonomous body of the National Assembly of Rectors (ANR),\footnote{The ANR was a public and autonomous body created by Law N� 23733, constituted by the rectors of the universities.} which had among its attributions to evaluate and authorize the operation of the new universities. In the case of public universities, beside the evaluation of CONAFU were required the creation law and the evaluation by the Ministry of Economy and Finance (MEF) on the capacity of the State to finance its operation.

In order to modernize the system and expand coverage, in 1996, Legislative Decree N� 882, Law for the Promotion of Investment in Education, was passed. Any natural or legal person was allowed to found and manage private universities with or without profit. CONAFU was in charge of the evaluation and authorization of their operation. 

Later, in 2001, under Law N� 27540, Law that regulates the creation of university affiliates and grants additional powers to the ANR, the creation of university affiliates outside the departmental scope of their creation is authorized under the ANR regulation. The absence of an entity that promotes academic quality policies generated the explosion of university affiliates without adequate infrastructure \cite{depaz2006universidad}. In 2005, through Law N� 28564, the creation of university affiliates was again prohibited.

Finally, in 2012, Law N� 29971 was enacted, a Law that established the moratorium on the creation of public and private universities and their subsidiaries for a period of 5 years. The purpose was to allow the university higher education policy to be reformed, towards one that establishes requirements for the creation and operation of duly accredited and certified universities that guarantee quality, research and relation with the country's development needs.
 
Regarding the legal framework that regulates the creation of non-university higher education institutes and schools in 1995-2012 period, there were a bunch of regulations at different levels; Legislative Decree N� 882 and Law N� 29394, Law of Institutes and Schools of Higher Education, are identified as legal frameworks of general application. Similar to the case of universities, Legislative Decree N� 882 enabled natural or legal persons to found and manage private institutes and higher schools with or without profit-making purposes, corresponding to the Ministry of Education to authorize their operation. Law N� 29394 published in 2009 stated that public institutes and schools were created with supreme resolution, with the prior favorable opinion of the corresponding regional education department and the MEF; while private institutes and schools were created at the initiative of individuals, whether natural or legal persons. In both cases, the authorization of operation and its revalidation required the approval of the Ministry of Education.\footnote{The authorization of operation of the institutes and School is granted for a period of no less than 3 years and no longer than 6 years.} Additionally, it is established that the scope of operation of the institutes and schools was the provincial one; any subsidiaries or annexes could not be created outside the scope of operation. 

\bibliographystyle{apalike}
\bibliography{bibliography}

\end{document}